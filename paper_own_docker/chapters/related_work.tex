\section{RELATED WORK}
\label{chap:related_work}

Previous studies have examined attack graph generation, primarily relative to computer networks \cite{ingols2006practical, sheyner2002automated, ritchey2000using, ou2006scalable}, where multiple machines are connected to each other and the Internet. %In these networks, an attacker performs multiple steps to achieve his goal, i.e., gaining privileges of a specific host. Attack graphs help in analyzing this behavior. Although attack graphs are useful, constructing them manually can be a cumbersome and time-consuming process. Tools that generate vulnerabilities of a specific host are available \cite{clair, artz2002netspa, farmer1990cops}. However previous works state that these tools alone are not enough to analyze the vulnerability of an entire network and that these tools in addition to network topology could solve this issue. This outcome is because different hosts are connected together and influence the outcome of an attack. Therefore, some teams have been working on developing systems that generate attack graphs automatically by using different approaches.
One early study of attack graph generation was conducted by Sheyner et al. using model checkers with a goal property \cite{sheyner2002automated}. Model checkers use computational logic to determine if a model is correct; otherwise, if the model is incorrect, the model checkers provide a counterexample. A collection of these counterexamples form an attack graph. Sheyner et al. stated that model checkers satisfy the monotonicity property to ensure termination. However, model checkers have a computational disadvantage. %In the example provided, NuSMV takes 2 hours to construct the attack graph with 5948 nodes and 68364 edges . As a result of this, more scalable approach was needed.
 Amman et al. extended this work with some simplifications and more efficient storage \cite{ritchey2000using}. Ou et al. used a logical attack graph \cite{ou2006scalable} and Ingols et al. \cite{ingols2006practical} used BFS algorithm to tackle the scalability issue. Ingols et al. discussed the redundancy of full and predictive graphs and modeled an attack graph as an MP graph with contentless edges and three node types. They used BFS technique to generate the attack graph. This approach provides faster results compared to using model checkers. For example, with this method, an MP graph with 8901 nodes and 23315 edges was constructed in 0.5 seconds. Aksu et al. conveyed a study on top of Ingols's system. They defined a specific set of precondition and postcondition rules and tested their correctness. Note that they used a machine learning approach in their evaluation \cite{aksu2018automated}.

Despite their increasing popularity, containers and microservice architectures have demonstrated serious security risks, primarily due to their connectivity requirements and lesser degree of encapsulation \cite{combe2016docker, dragoni2017microservices}. To the best of our knowledge, no previous study has been conducted relative to attack graph generation for Docker containers. Similar to computer networks, microservice architectures have a container topology and tools for container analysis. Containers in our model correspond to hosts, and a connection between hosts translates to communication between containers. 

In summary, our contribution is using attack graph generation as part of DevOps practices and providing tool support for this concept. To that end, we have extended the work of Ingols \cite{ingols2006practical} and Aksu \cite{aksu2018automated} in conjunction with the Clair OS to generate attack graphs for microservice architectures. 

