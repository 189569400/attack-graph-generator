\section{RELATED WORK}
Previous work has dealt with attack graph generation, mainly in computer networks 
%\todo{please cite them}
, where multiple machines are connected to each other and the Internet. In these networks an attacker performs multiple steps to achieve his goal, i.e., gaining privileges of a specific host. Attack graphs help in analyzing this behavior. Although attack graphs are useful, constructing them manually can be a cumbersome and time-consuming process
%\todo{elaborate a bit why; scalability, expertise, completeness.. }
. Tools that generate vulnerabilities of specific host are available. However previous works have state that these tools alone are not enough to analyze the vulnerability 
%\todo{mention why coz they can be connected ..} 
of an entire network and that these tools in addition to network topology could solve this issue. Therefore, some teams have been  working on developing systems that generate attack graphs automatically by using different approaches.

One of the earlier works in attack graph generation was done by Sheyner with using model checkers with goal property.\cite{ingols2006practical} %\todo{space citation then the dot, e.g., property [5].}

Model checkers use computational logic to check if a model is correct, and otherwise, they provide a counter example. A collection of these counter examples form an attack graph. They state that model checkers satisfy a monotonicity property in order to ensure termination. However model checkers have a computational disadvantage. In the example provided, NuSMV takes 2 hours to construct the attack graph with 5948 nodes and 68364 edges(Ingols reference) As a result of this, more scalable approach was needed. Amman et al. extends this work with some simplifications and more efficient storage \cite{ritchey2000using}.

Ou et al. use logical attack graph \cite{ou2006scalable} and Ingols \cite{ingols2006practical} et al. use breath first search algorithm in order to tackle the scalability issue. Ingols discusses the redundancy Full and Predictive graphs and model an attack graph as a MP graph with contentless edges and 3 types of nodes
%\todo{please rephrase this sentence to clarify it what is redundanccy full, define MP}
. They use breath first search technique for generating the attack graph. This approach provides faster results in comparison to using model checkers. An MP graph of 8901 nodes and 23315 edges is constructed in 0.5 seconds.

Aksu et al. builds on top of Ingols's system and evaluate a set of rule pre- and postcondtions in generating attacks \cite{aksu2018automated}. They define a specific test of pre- and postcondition rules and test their correctness. In their evaluation, they use machine learning approach.

Containers and microservice architectures, despite their ever-growing popularity, have shown somewhat bigger security risks, mostly because of their bigger need of connectivity and lesser degree of encapsulation(Reference).To the best of our knowledge, there is no work that has been done so far in the area of attack graph generation for docker containers. Therefore we extend the work from Ingols \cite{ingols2006practical} and Aksu \cite{aksu2018automated} in conjuction to Clair OS to generate attack graphs for microservice architectures. Similar to computer networks, microservice architectures have a container topology and tools for analysis of containers. Containers in our model correspond to hosts, and a connection between hosts translates to a communication between containers. 
%\todo{in general good story line, some language work is needed later-on}

