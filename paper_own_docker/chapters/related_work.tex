\section{RELATED WORK}
\label{chap:related_work}

Previous work has dealt with attack graph generation, mainly in computer networks \cite{ingols2006practical, sheyner2002automated, ritchey2000using, ou2006scalable}, where multiple machines are connected to each other and the Internet. %In these networks, an attacker performs multiple steps to achieve his goal, i.e., gaining privileges of a specific host. Attack graphs help in analyzing this behavior. Although attack graphs are useful, constructing them manually can be a cumbersome and time-consuming process. Tools that generate vulnerabilities of a specific host are available \cite{clair, artz2002netspa, farmer1990cops}. However previous works state that these tools alone are not enough to analyze the vulnerability of an entire network and that these tools in addition to network topology could solve this issue. This outcome is because different hosts are connected together and influence the outcome of an attack. Therefore, some teams have been working on developing systems that generate attack graphs automatically by using different approaches.
One of the earlier works in attack graph generation was done by Sheyner et al. by using model checkers with goal property \cite{sheyner2002automated}. Model checkers use computational logic to check if a model is correct, and otherwise, they provide a counterexample. A collection of these counterexamples form an attack graph. They state that model checkers satisfy a monotonicity property in order to ensure termination. However, model checkers have a computational disadvantage. In the example provided, NuSMV takes 2 hours to construct the attack graph with 5948 nodes and 68364 edges \cite{sheyner2002automated}. As a result of this, more scalable approach was needed. Amman et al. extend this work with some simplifications and more efficient storage \cite{ritchey2000using}. Ou et al. use logical attack graph \cite{ou2006scalable} and Ingols \cite{ingols2006practical} et al. use Breadth-first search algorithm in order to tackle the scalability issue. Ingols et al. discuss the redundancy Full and Predictive graphs and model an attack graph as an MP graph with contentless edges and 3 types of nodes. They use Breath-first search technique for generating the attack graph. This approach provides faster results in comparison to using model checkers. An MP graph of 8901 nodes and 23315 edges is constructed in 0.5 seconds. Aksu et al. build on top of Ingols's system and evaluate a set of rule pre- and postconditions in generating attacks. They define a specific test of pre- and postcondition rules and test their correctness. In their evaluation, they use a machine learning approach \cite{aksu2018automated}.

Containers and microservice architectures, despite their ever-growing popularity, have shown somewhat bigger security risks, mostly because of their bigger need of connectivity and a lesser degree of encapsulation \cite{combe2016docker, dragoni2017microservices}. To the best of our knowledge, there is no work that has been done so far in the area of attack graph generation for Docker containers. Similar to computer networks, microservice architectures have a container topology and tools for analysis of containers. Containers in our model correspond to hosts, and a connection between hosts translates to a communication between containers. Therefore we extended the work from Ingols \cite{ingols2006practical} and Aksu \cite{aksu2018automated} in conjunction to Clair OS \cite{clair} to generate attack graphs for microservice architectures. 

