\subsection{Microservices}
\label{chap:microservices}

As real-world software grows in size, there is an ever growing need to decompose it into an organized structure to promote scaling, reuse and readability. A software application whose modules cannot be executed independently is called a monolith. Monolithic systems  are characterized by tight coupling, vertical scaling and strong dependence \cite{microservicesfrowler}. Service Oriented Architecture (SOA) addresses these issues by restructuring its elements into components that provide services which are used by other entities through a networking protocol \cite{papazoglou2003service}. However, in a typical SOA, the services are monolithic which gives rise to the concept of microservices in order to provide an even more fine-grained task separation \cite{ahmadvand2016requirements}. The novel term "microservices" was first introduced in 2011 at an architectural workshop in order to propose a common term for the explorations of multiple researchers \cite{dragoni2017microservices, microservicesfrowler}. In the microservices paradigm, multiple services are split into very basic units which are task oriented. According to Dragoni et al. a microservice is a cohesive, independent process interacting via messages. These microservices constitute as a distributed architecture called a microservice architecture \cite{dragoni2017microservices}. Microservice architectures benefit us with the advantage of having more heterogeneous technologies, cheaper scaling, resilience, organizational alignment, and composability \cite{newman2015building}. However, they add an additional complexity and have a wider attack surface as the need of many services to communicate with each other and third-party software increases \cite{combe2016docker, dragoni2017microservices}. While microservices are an architectural principle, container technology has emerged in cloud computing to provide a lightweight virtualization mechanism. This technology enables microservices to be packaged and orchestrated through the Cloud \cite{pahl2016microservices}.

