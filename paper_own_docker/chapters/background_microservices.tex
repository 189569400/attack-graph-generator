\subsection{Microservices}
\label{chap:microservices}

As real world software grows bigger, there is an ever growing need to decompose it into an organized structure to promote scaling, reuse and readability. A software application whose modules cannot be executed independently is called a monolith. Monoliths are characterized by tight coupling, vertical scaling and strong dependence. Service Oriented Architecture(SOA) addresses these issues by restructuring its elements into components that provide services which are used by other entities through a networking protocol \cite{papazoglou2003service}. However in a typical SOA, the services are monolithic which gives rise to microservices in order to provide an even more fine-grained task separation \cite{ahmadvand2016requirements}.
In microservices paradigm, multiple services are split into very basic units which are task oriented. According to Dragoni et al. a microservice is a cohesive, independent process interacting via messages. There microservices constitute a distributed architecture called a microservice architecture \cite{dragoni2017microservices}. Microservice architectures provide us with the advantage of having more heterogenious technologies, cheaper scaling, resilience, organizational alignment, and composability among other benefits \cite{newman2015building}. However, they add an additional complexity and have a wider attack surface as the need of many services to communicate with each other and third party software increases. 

