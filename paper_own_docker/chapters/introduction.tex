\section{INTRODUCTION}

%intro about the new paradiagm of mirocservices its benefits and new oppurnties, maybe its utilization in secuirty/ safety critical systems
Microservices, as a new approach to manage the complexity of modern applications, are increasingly adopted in real-world systems. The new architectural style 
follows the Unix fundamental principles  of decomposing systems into small programs \cite{Unix ideas [115]:} that each fulfills only one cohesive task and can work together using universal interfaces. Each program is a microservice that is designed, developed, tested,  deployed, and scaled independently \cite{[Microservices. Martin Fowler(2015)]}. The smaller decoupled services have a positive impact on some system qualities like scalability, fault isolation, and technology heterogeneity \cite{}. However, clearly, other qualities like the network utilization, and the security are negatively affected \cite{ahmadvand2016requirements}.  Balancing the trades-off among these factors derive the decision of using microservices in industry. That said, a non exhaustive list \footnote{\url{https://microservices.io/articles/whoisusingmicroservices.html}} shows a significant shift by many enterprises across different domains towards using microservice-based architecture. This shift is  motivated mainly by the demanding requirements of scalability, time to market, and better optimization of development efforts. We see microservice-based systems in domains of video streaming, social networks, logistics, internet of things \cite{internet}, and smart cities \cite{smartcity}. 

% the new tools or eco system of this style, especially container based deployment... the benefits of it and the challanges brought by it, like open source utilization and scurity overhead brought by networks and docker security issues. 
Microservices have popularized two main concepts in the software engineering community. The first is the \textit{container-based deployment}, in which the new small services are shipped and deployed in lightweight containers. As a result the systems are deployed as networks of communicating micorservices. Each microservice achieves an independent task. 
% threat modeling as a way to systmatically analyze threats... its benefits and obstecales, ag as an example especially for newtorked systems... the mapping between container based and network applications... and the progress there. 

% the problem: summing it up and the solution
% the potentisl benefit in the scope of devops practices and automation (automation should be mentioned before)
% contribution  focus on docker and MS in comparison with networks and normal systems and maybe the github
% structure 


Attack graphs are a popular way of examining security aspects of network. They help security analysts to carefully analyze a system connection and detect the most vulnerable parts of the system. An attack graph depicts the actions that an attacker uses in order to reach his goal.  

Statistics. 
https://banyanops.com/blog/analyzing-docker-hub/
https://blog.acolyer.org/2017/04/03/a-study-of-security-vulnerabilities-on-docker-hub/
http://dance.csc.ncsu.edu/papers/codaspy17.pdf \cite{shu2017study}

They do not offer any performance comparison between different topologies. 

In this work, we first get familiar with the basic attack graph and microservice terminology in Section \ref{chap:background}, then present the architecture of our system in Section \ref{chap:method}, perform evaluation in Section \ref{chap:eval} and at the end present what others did in the area in section \ref{chap:related_work}, a conclusion in Section \ref{chap:conclusion} and future work directions in Section \ref{chap:future_work}.

