\section{INTRODUCTION}

%intro about the new paradiagm of mirocservices its benefits and new oppurnties, maybe its utilization in secuirty/ safety critical systems
Microservices, as a new approach to manage the complexity of modern applications, are increasingly adopted in real-world systems. The new architectural style 
follows the Unix fundamental principles  of decomposing systems into small programs \cite{Unix ideas [115]:} that each fulfills only one cohesive task and can work together using universal interfaces. Each program is a microservice that is designed, developed, tested,  deployed, and scaled independently \cite{[Microservices. Martin Fowler(2015)]}. The smaller decoupled services have a positive impact on some system qualities like scalability, fault isolation, and technology heterogeneity \cite{}. However, clearly, other qualities like the network utilization, and the security are negatively affected \cite{ahmadvand2016requirements}.  Balancing the trades-off among these factors derive the decision of using microservices in industry. That said, a non exhaustive list \footnote{\url{https://microservices.io/articles/whoisusingmicroservices.html}} shows a significant shift by many enterprises across different domains towards using microservice-based architecture. This shift is  motivated mainly by the demanding requirements of scalability, time to market, and better optimization of development efforts. We see microservice-based systems in domains of video streaming, social networks, logistics, internet of things \cite{internet}, and smart cities \cite{smartcity}. 

% the new tools or eco system of this style, especially container based deployment... the benefits of it and the challanges brought by it, like open source utilization and scurity overhead brought by networks and docker security issues. 
The utilization of microservices have popularized two main concepts in the software engineering community. The first is the \textit{container-based deployment}, in which the new small services are shipped and deployed in containers \cite{bibid}. As a result, the systems are deployed as networks of communicating micorservices. For their lightweight and operating-system level virtualization \cite{what is the container hype}, the containerization frameworks like \textit{Docker} \cite{bibid}, are a high performance alternative to hypervisors \cite{nw}. The second often-used concept in the domain of microservices development is DevOps \cite{bibid}. DevOps enable practices in which full automation of the deployment process is achieved. In the course of this, end-to-end automated packaging and deployment  is a vital part of microservices development. In addition to the agility and optimization brought by the two concepts, major concerns around their impact on security \cite{ahmadvand2016requirements}. The increasing communication end-points among the microservices \cite{ahmadvand2016requirements}, the potentially growing number of vulnerabilities emerging from open-source DevOps tools and third-party frameworks distributed by docker hub \cite{shu2017study,gummaraju2015over}, and container isolation. 


 In this paper, we tackle the problem of analyzing the security of the container networks using threat models \cite{kordysurvey}. Following the Devops mentality, we propose an automated method that can be integrated into continuous delivery systems to generate attack graphs.

% threat modeling as a way to systmatically analyze threats... its benefits and obstecales, ag as an example especially for newtorked systems... the mapping between container based and network applications... and the progress there. 


Previous work has dealt with attack graph generation, mainly in computer networks \cite{ingols2006practical, sheyner2002automated, ritchey2000using, ou2006scalable}, where multiple machines are connected to each other and the Internet. In these networks, an attacker performs multiple steps to achieve his goal, i.e., gaining privileges of a specific host. Attack graphs help in analyzing this behavior. Although attack graphs are useful, constructing them manually can be a cumbersome and time-consuming process. Tools that generate vulnerabilities of a specific host are available \cite{clair, artz2002netspa, farmer1990cops}. However previous works state that these tools alone are not enough to analyze the vulnerability of an entire network and that these tools in addition to network topology could solve this issue. This outcome is because different hosts are connected together and influence the outcome of an attack. Therefore, some teams have been working on developing systems that generate attack graphs automatically by using different approaches.


% the problem: summing it up and the solution
% the potentisl benefit in the scope of devops practices and automation (automation should be mentioned before)
% contribution  focus on docker and MS in comparison with networks and normal systems and maybe the github
% structure 


Attack graphs are a popular way of examining security aspects of network. They help security analysts to carefully analyze a system connection and detect the most vulnerable parts of the system. An attack graph depicts the actions that an attacker uses in order to reach his goal.  

Statistics. 
https://banyanops.com/blog/analyzing-docker-hub/
https://blog.acolyer.org/2017/04/03/a-study-of-security-vulnerabilities-on-docker-hub/
http://dance.csc.ncsu.edu/papers/codaspy17.pdf \cite{shu2017study}

They do not offer any performance comparison between different topologies. 

In this work, we first get familiar with the basic attack graph and microservice terminology in Section \ref{chap:background}, then present the architecture of our system in Section \ref{chap:method}, perform evaluation in Section \ref{chap:eval} and at the end present what others did in the area in section \ref{chap:related_work}, a conclusion in Section \ref{chap:conclusion} and future work directions in Section \ref{chap:future_work}.

