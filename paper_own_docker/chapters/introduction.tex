\section{INTRODUCTION}

%intro about the new paradiagm of mirocservices its benefits and new oppurnties, maybe its utilization in secuirty/ safety critical systems
Microservices, a recent approach to managing the complexity of modern applications, are increasingly being adopted in real-world systems. Microservice architectures follow the fundamental principle of Unix, i.e., systems are decomposed into small programs \cite{wolff2016microservices}, each performing a single cohesive task. However, such programs can work together via universal interfaces, where each program is a microservice that is designed, developed, tested,  deployed, and scaled independently \cite{microservicesfrowler}. Smaller decoupled services have a positive impact on some system qualities, such as scalability, fault isolation, and technology heterogeneity \cite{newman2015building}; however, other qualities, such as network utilization and security, can be affected negatively \cite{ahmadvand2016requirements}. The decision to use microservices in industrial applications must consider the tradeoffs among these factors. That said, a list \footnote{\url{https://microservices.io/articles/whoisusingmicroservices.html}} of companies from different domains that use microservice-based architecture indicates a significant shift towards their use. This shift is primarily motivated by the demanding requirements of scalability, time to market, and improving development optimization. Microservice-based systems can be seen in various domains, such as video streaming, social networks, logistics, the Internet of Things \cite{butzin2016microservices}, smart cities \cite{krylovskiy2015designing}, and security-critical systems \cite{fetzer2016building}. 

% the new tools or eco system of this style, especially container based deployment... the benefits of it and the challanges brought by it, like open source utilization and scurity overhead brought by networks and docker security issues. 
The utilization of microservices has popularized two main concepts in the software engineering community. The first is \textit{container-based deployment}, where new small services are shipped and deployed in containers \cite{jaramillo2016leveraging}. As a result, such systems are deployed as networks of communicating microservices. Due to their lightweight and operating-system level virtualization \cite{bottomely}, containerization frameworks, such as \textit{Docker} \cite{cerny2018contextual}, have become a high-performance alternative to hypervisors \cite{kratzke2017microservices}. The second concept in microservices development is \textit{DevOps} \cite{cerny2018contextual}, which enable practices that can fully automate the deployment process. Here, end-to-end automated packaging and deployment is a vital component of microservice development. In addition to the agility and optimization realized by these two concepts, significant concerns have been raised about their security \cite{ahmadvand2016requirements}. These concerns are motivated by the increasing number of communication endpoints among microservices, the potentially increasing number of vulnerabilities emerging from open-source DevOps tools and third-party frameworks distributed by Docker hub \cite{shu2017study,gummaraju2015over}, and weaker isolation (compared to hypervisor-based virtualization) between hosts and containers because all containers share the same kernel \cite{bottomely, bui2015analysis}. In this paper, we address the problem of analyzing the security of container networks using threat models \cite{kordy2014dag}. Following the DevOps spirit, we propose an automated method that can be integrated into continuous delivery systems to generate attack graphs.


% threat modeling as a way to systmatically analyze threats... its benefits and obstecales, ag as an example especially for newtorked systems... the mapping between container based and network applications... and the progress there. 

Security threat models are widely used to assess threats to a system \cite{kordy2014dag}. They appeal to practitioners because they provide visual presentations of possible attack paths in a system. They also appeal to scientists, because they are well formalized (syntax and semantics) \cite{mauw2005foundations,jha2002two}. Such formalism enables quantitative and qualitative analysis of the risk, cost, and likelihood of attacks, which affect defense strategies. In computer networks, attack graphs \cite{sheyner2002automated,ou2006scalable} are the dominant threat model used to inspect the security aspects of a network. They help analysts carefully analyze system connections and detect the most vulnerable parts of a system. An attack graph depicts the actions an attacker may use to reach their goal. Typically, experts (e.g., red teams) construct attack graphs manually; however, this manual process is time-consuming, error-prone, and does not address the complexity of modern infrastructures.  %Within the last 15 years, there have been some proposals of approaches [e.g., 5-9] to automate the generation of attack graphs and attack trees in the domain of network security.


% the problem: summing it up and the solution
% the potentisl benefit in the scope of devops practices and automation (automation should be mentioned before)
% contribution  focus on docker and MS in comparison with networks and normal systems and maybe the github
Previous studies have dealt with automatic attack graph generation for computer networks \cite{ingols2006practical, sheyner2002automated, ou2006scalable}. In these networks, an attacker performs multiple steps to achieve his goal, e.g., gaining the privileges of a specific host. Tools that scan the vulnerabilities of a particular host are available \cite{farmer1990cops}; however, they are insufficient to analyze the security of an entire network and the possible composition of various vulnerability exploitation as an attack path \cite{sheyner2002automated}. 

To the best of our knowledge, automated attack graph generation for microservice architectures has not been examined by previous studies. To that end, we extend advancements made in the computer networks field to the microservice domain. The primary contributions of this paper are summarized as follows.
\begin{itemize}
	\item We propose attack graphs as a new artifact for continuous delivery systems. We present an approach based on methods from computer networks to automatically generate attack graphs for microservice-based architectures deployed as containers.   
	\item We present the technical details of an extensible tool that implements the proposed approach. Note that this tool is available online \footnote{The link is omitted for anonymization  purpose}.
	\item We provide an empirical evaluation of the efficiency of the proposed approach relative to generating attack graphs for real-world systems.
\end{itemize}



% structure 

The remainder of this paper is organized as follows. We introduce preliminaries in Section \ref{chap:background}. Then, we present the proposed approach in Section \ref{chap:method}. An evaluation of the proposed approach is given in Section \ref{chap:eval}. We discuss related work in Section \ref{chap:related_work}. Conclusions and suggestions for future work are discussed in Section \ref{chap:conclusion}.

