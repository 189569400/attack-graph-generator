\subsection{Attack Graphs}
\label{chap:attack_graphs}
In this work we model an attack graph as a sequence of atomic attacks. Each atomic attack represents a transition from a component(with its privilege) to either the same component with a higher privilege or to a neighbor component with a new privilege. A goal of an attacker would be to perform multiple atomic attacks to reach the desired goal container. For example, let us suppose that an attacker wants to have access to a database of a certain website. In order to reach the database, he has to pass other containers between him and the database. He does that by finding a vulnerability to exploit and gives access to the container's neighbors. After a successful exploitation of multiple intermediate containers, he finally has access to the database and its content. Even though attack graphs model the attacker scenario from an attackers perspective, they are of crucial importance in computer security. For example, a system administrator would be interested to have an overview of the attack paths that an attacker could exploit, in order to harden the security of a given enterprise system.

Furthermore, in order for atomic attacks to be performed, certain preconditions have to be ensured. Preconditions are privilege levels that are required so that a vulnerability can be exploited. When an atomic attack is successfully executed, a postcondition is obtained. Postconditions are privileges acquired as a result of a successful attack. 

- mention monotonicity