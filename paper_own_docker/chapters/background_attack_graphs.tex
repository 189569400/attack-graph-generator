\subsection{Attack Graphs}
\label{chap:attack_graphs}
\textbf{TODO}  why do we use them\\

The definition of attack graphs may vary but it is essentially a directed graph that consists of nodes and edges with various representations. In this subsection, we first look at a few examples of how others define attacks graphs and at the end present the model of an attack graph that we use in this work.

Seyner et al. define an attack graph as a tuple of states, transitions between the states, initial state and success states. An initial state represents the state from where the attacker starts the attack and through a chain of atomic attacks tries to reach one of the success states \cite{sheyner2002automated}.

Ou et al. introduce the notion of a logical attack graph. A logical attack graph is a bipartite directed graph that consists of two kinds of nodes: fact nodes and derivation nodes. Each fact node is labeled with a logical statement in the form of a predicate applied to its arguments, while each derivation node is labeled with an interaction rule that is used for the derivation step. The edges in the graph represent the "depends on" relation \cite{ou2006scalable}.

Ingols et al. make a distinction between full, predictive and multiple-prerequisite (MP) attack graphs. Full graph is a directed acyclic graph that consists of nodes that represent hosts and edges that represent vulnerability instances. Predictive attack graphs use the same representation as full attack graphs with the only difference lying in the constraint of when the edges are added to the attack graph. These graphs are generally smaller than full graphs. MP attack is an attack graph with as contentless edges and three node type: state nodes, vulnerability instance nodes and prerequisite nodes \cite{ingols2006practical}.

In our work we define attack graph to be a directed acyclic graph with a set of nodes and edges similar to the full graph representation of Ingols et al. \cite{ingols2006practical}. As an expansion to this model, a node represents a state of a host with its current privilege. An edge represents a successful transition between two such hosts. We can think of an edge as a successful vulnerability exploitation which is initiated from a host with a required privilege to another or the same host with the newly gained privilege as a result of the vulnerability exploitation.