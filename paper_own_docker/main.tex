%%%%%%%%%%%%%%%%%%%%%%%%%%%%%%%%%%%%%%%%%%%%%%%%%%%%%%%%%%%%%%%%%%%%%%%%%%%%%%%%
%2345678901234567890123456789012345678901234567890123456789012345678901234567890
%        1         2         3         4         5         6         7         8

\documentclass[letterpaper, 10 pt, conference]{ieeeconf}  % Comment this line out
\usepackage[options]{algorithm2e}                                     % if you need a4paper
%\documentclass[a4paper, 10pt, conference]{ieeeconf}      % Use this line for a4
                                                          % paper

\IEEEoverridecommandlockouts                              % This command is only
                                                          % needed if you want to
                                                          % use the \thanks command
\overrideIEEEmargins
% See the \addtolength command later in the file to balance the column lengths
% on the last page of the document



% The following packages can be found on http:\\www.ctan.org
%\usepackage{graphics} % for pdf, bitmapped graphics files
%\usepackage{epsfig} % for postscript graphics files
%\usepackage{mathptmx} % assumes new font selection scheme installed
%\usepackage{times} % assumes new font selection scheme installed
%\usepackage{amsmath} % assumes amsmath package installed
%\usepackage{amssymb}  % assumes amsmath package installed

\title{\LARGE \bf
Attack Graph Generation for Micro-service
Architecture
}

%\author{ \parbox{3 in}{\centering Huibert Kwakernaak*
%         \thanks{*Use the $\backslash$thanks command to put information here}\\
%         Faculty of Electrical Engineering, Mathematics and Computer Science\\
%         University of Twente\\
%         7500 AE Enschede, The Netherlands\\
%         {\tt\small h.kwakernaak@autsubmit.com}}
%         \hspace*{ 0.5 in}
%         \parbox{3 in}{ \centering Pradeep Misra**
%         \thanks{**The footnote marks may be inserted manually}\\
%        Department of Electrical Engineering \\
%         Wright State University\\
%         Dayton, OH 45435, USA\\
%         {\tt\small pmisra@cs.wright.edu}}
%}

\author{Stevica Bozhinoski$^{1}$ and Amjad Ibrahim$^{2}$% <-this % stops a space
}

\begin{document}



\maketitle
\thispagestyle{empty}
\pagestyle{empty}


%%%%%%%%%%%%%%%%%%%%%%%%%%%%%%%%%%%%%%%%%%%%%%%%%%%%%%%%%%%%%%%%%%%%%%%%%%%%%%%%
\begin{abstract}


\end{abstract}


%%%%%%%%%%%%%%%%%%%%%%%%%%%%%%%%%%%%%%%%%%%%%%%%%%%%%%%%%%%%%%%%%%%%%%%%%%%%%%%%
\section{INTRODUCTION}

\section{RELATED WORK}

What makes us different to previous work:
- To the best of our knowledge, there is no work that has been done for attack graph generation for docker containers.
- The paper(Automated Generation of Attack Graphs Using NVD) does not provide an example/ or clearly define the attack graph generation process. Their main focus is to evaluate attack rules. They dont use Clair as well. They do not offer any performance comparison between different topologies. 

\section{ARCHITECTURE}

\subsubsection{Breadth-first search}

We use a modification of the breadth first search algorithm to find the nodes and the edges.
\begin{algorithm}
	\SetAlgoLined
	\KwData{goal\_container, topology, container\_exploitability}
	\KwResult{nodes, edges}
	nodes, edges, queue, passed\_nodes = list(), dict\{\}, Queue(), []\;
	queue.put(goal\_container)\;
	
	nodes = get\_nodes()\;
	\While{! queue.isEmpty()}{
		ending\_node = queue.get()\;
		passed\_nodes[ending\_node] = True\;
		cont\_exp\_end = container\_exploitability[ending\_node]\;
		neighbours = topology[ending\_node]\;
	    \For{neighbour in neighbours}{
	    	\If{neighbour == "outside"}{
	    		
	    			edges.append(create\_edges())\;
	    			continue\;

	    		}
	        \If{! passed\_nodes[neighbour]}
	        {queue.put(neighbour)\;}
            \If{neighbour == goal\_container}{
            	continue\;}
            edges.append(create\_edges())
	    	}
	
	}

\caption{Breadth-first search algorithm for generating an attack graph.}
\end{algorithm}

The algorithm requires the goal container, the topology and a dictionary of the exploitable vulnerabilities as an input and the output is made up of the nodes and the edges that make the attack graph. 
The algorithm first initializes the nodes, edges, queue and the passed nodes. Afterwards it generates the nodes which are a combination of the image name and the exploitable vulnerability.
Then into a while loop we iterate through every node, check its neighbours and add the edges. If the neighbour was not passed, then we add it to the queue. The algorithm terminates when the queue is empty.
\section{EXPERIMENTS}
\section{CONCLUSION}




\addtolength{\textheight}{-12cm}   % This command serves to balance the column lengths
                                  % on the last page of the document manually. It shortens
                                  % the textheight of the last page by a suitable amount.
                                  % This command does not take effect until the next page
                                  % so it should come on the page before the last. Make
                                  % sure that you do not shorten the textheight too much.

%%%%%%%%%%%%%%%%%%%%%%%%%%%%%%%%%%%%%%%%%%%%%%%%%%%%%%%%%%%%%%%%%%%%%%%%%%%%%%%%



%%%%%%%%%%%%%%%%%%%%%%%%%%%%%%%%%%%%%%%%%%%%%%%%%%%%%%%%%%%%%%%%%%%%%%%%%%%%%%%%



%%%%%%%%%%%%%%%%%%%%%%%%%%%%%%%%%%%%%%%%%%%%%%%%%%%%%%%%%%%%%%%%%%%%%%%%%%%%%%%%

\section*{ACKNOWLEDGMENT}





%%%%%%%%%%%%%%%%%%%%%%%%%%%%%%%%%%%%%%%%%%%%%%%%%%%%%%%%%%%%%%%%%%%%%%%%%%%%%%%%


\begin{thebibliography}{99}

\bibitem{c1} Merkel, Dirk. "Docker: lightweight linux containers for consistent development and deployment." Linux Journal 2014.239 (2014): 2.







\end{thebibliography}




\end{document}
